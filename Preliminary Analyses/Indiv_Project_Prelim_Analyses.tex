% Options for packages loaded elsewhere
\PassOptionsToPackage{unicode}{hyperref}
\PassOptionsToPackage{hyphens}{url}
%
\documentclass[
]{article}
\title{Individual Project Preliminary Analyses}
\author{Hannah Rodgers}
\date{4/27/2022}

\usepackage{amsmath,amssymb}
\usepackage{lmodern}
\usepackage{iftex}
\ifPDFTeX
  \usepackage[T1]{fontenc}
  \usepackage[utf8]{inputenc}
  \usepackage{textcomp} % provide euro and other symbols
\else % if luatex or xetex
  \usepackage{unicode-math}
  \defaultfontfeatures{Scale=MatchLowercase}
  \defaultfontfeatures[\rmfamily]{Ligatures=TeX,Scale=1}
\fi
% Use upquote if available, for straight quotes in verbatim environments
\IfFileExists{upquote.sty}{\usepackage{upquote}}{}
\IfFileExists{microtype.sty}{% use microtype if available
  \usepackage[]{microtype}
  \UseMicrotypeSet[protrusion]{basicmath} % disable protrusion for tt fonts
}{}
\usepackage{xcolor}
\IfFileExists{xurl.sty}{\usepackage{xurl}}{} % add URL line breaks if available
\IfFileExists{bookmark.sty}{\usepackage{bookmark}}{\usepackage{hyperref}}
\hypersetup{
  pdftitle={Individual Project Preliminary Analyses},
  pdfauthor={Hannah Rodgers},
  hidelinks,
  pdfcreator={LaTeX via pandoc}}
\urlstyle{same} % disable monospaced font for URLs
\usepackage[margin=1in]{geometry}
\usepackage{color}
\usepackage{fancyvrb}
\newcommand{\VerbBar}{|}
\newcommand{\VERB}{\Verb[commandchars=\\\{\}]}
\DefineVerbatimEnvironment{Highlighting}{Verbatim}{commandchars=\\\{\}}
% Add ',fontsize=\small' for more characters per line
\usepackage{framed}
\definecolor{shadecolor}{RGB}{248,248,248}
\newenvironment{Shaded}{\begin{snugshade}}{\end{snugshade}}
\newcommand{\AlertTok}[1]{\textcolor[rgb]{0.94,0.16,0.16}{#1}}
\newcommand{\AnnotationTok}[1]{\textcolor[rgb]{0.56,0.35,0.01}{\textbf{\textit{#1}}}}
\newcommand{\AttributeTok}[1]{\textcolor[rgb]{0.77,0.63,0.00}{#1}}
\newcommand{\BaseNTok}[1]{\textcolor[rgb]{0.00,0.00,0.81}{#1}}
\newcommand{\BuiltInTok}[1]{#1}
\newcommand{\CharTok}[1]{\textcolor[rgb]{0.31,0.60,0.02}{#1}}
\newcommand{\CommentTok}[1]{\textcolor[rgb]{0.56,0.35,0.01}{\textit{#1}}}
\newcommand{\CommentVarTok}[1]{\textcolor[rgb]{0.56,0.35,0.01}{\textbf{\textit{#1}}}}
\newcommand{\ConstantTok}[1]{\textcolor[rgb]{0.00,0.00,0.00}{#1}}
\newcommand{\ControlFlowTok}[1]{\textcolor[rgb]{0.13,0.29,0.53}{\textbf{#1}}}
\newcommand{\DataTypeTok}[1]{\textcolor[rgb]{0.13,0.29,0.53}{#1}}
\newcommand{\DecValTok}[1]{\textcolor[rgb]{0.00,0.00,0.81}{#1}}
\newcommand{\DocumentationTok}[1]{\textcolor[rgb]{0.56,0.35,0.01}{\textbf{\textit{#1}}}}
\newcommand{\ErrorTok}[1]{\textcolor[rgb]{0.64,0.00,0.00}{\textbf{#1}}}
\newcommand{\ExtensionTok}[1]{#1}
\newcommand{\FloatTok}[1]{\textcolor[rgb]{0.00,0.00,0.81}{#1}}
\newcommand{\FunctionTok}[1]{\textcolor[rgb]{0.00,0.00,0.00}{#1}}
\newcommand{\ImportTok}[1]{#1}
\newcommand{\InformationTok}[1]{\textcolor[rgb]{0.56,0.35,0.01}{\textbf{\textit{#1}}}}
\newcommand{\KeywordTok}[1]{\textcolor[rgb]{0.13,0.29,0.53}{\textbf{#1}}}
\newcommand{\NormalTok}[1]{#1}
\newcommand{\OperatorTok}[1]{\textcolor[rgb]{0.81,0.36,0.00}{\textbf{#1}}}
\newcommand{\OtherTok}[1]{\textcolor[rgb]{0.56,0.35,0.01}{#1}}
\newcommand{\PreprocessorTok}[1]{\textcolor[rgb]{0.56,0.35,0.01}{\textit{#1}}}
\newcommand{\RegionMarkerTok}[1]{#1}
\newcommand{\SpecialCharTok}[1]{\textcolor[rgb]{0.00,0.00,0.00}{#1}}
\newcommand{\SpecialStringTok}[1]{\textcolor[rgb]{0.31,0.60,0.02}{#1}}
\newcommand{\StringTok}[1]{\textcolor[rgb]{0.31,0.60,0.02}{#1}}
\newcommand{\VariableTok}[1]{\textcolor[rgb]{0.00,0.00,0.00}{#1}}
\newcommand{\VerbatimStringTok}[1]{\textcolor[rgb]{0.31,0.60,0.02}{#1}}
\newcommand{\WarningTok}[1]{\textcolor[rgb]{0.56,0.35,0.01}{\textbf{\textit{#1}}}}
\usepackage{graphicx}
\makeatletter
\def\maxwidth{\ifdim\Gin@nat@width>\linewidth\linewidth\else\Gin@nat@width\fi}
\def\maxheight{\ifdim\Gin@nat@height>\textheight\textheight\else\Gin@nat@height\fi}
\makeatother
% Scale images if necessary, so that they will not overflow the page
% margins by default, and it is still possible to overwrite the defaults
% using explicit options in \includegraphics[width, height, ...]{}
\setkeys{Gin}{width=\maxwidth,height=\maxheight,keepaspectratio}
% Set default figure placement to htbp
\makeatletter
\def\fps@figure{htbp}
\makeatother
\setlength{\emergencystretch}{3em} % prevent overfull lines
\providecommand{\tightlist}{%
  \setlength{\itemsep}{0pt}\setlength{\parskip}{0pt}}
\setcounter{secnumdepth}{-\maxdimen} % remove section numbering
\newlength{\cslhangindent}
\setlength{\cslhangindent}{1.5em}
\newlength{\csllabelwidth}
\setlength{\csllabelwidth}{3em}
\newlength{\cslentryspacingunit} % times entry-spacing
\setlength{\cslentryspacingunit}{\parskip}
\newenvironment{CSLReferences}[2] % #1 hanging-ident, #2 entry spacing
 {% don't indent paragraphs
  \setlength{\parindent}{0pt}
  % turn on hanging indent if param 1 is 1
  \ifodd #1
  \let\oldpar\par
  \def\par{\hangindent=\cslhangindent\oldpar}
  \fi
  % set entry spacing
  \setlength{\parskip}{#2\cslentryspacingunit}
 }%
 {}
\usepackage{calc}
\newcommand{\CSLBlock}[1]{#1\hfill\break}
\newcommand{\CSLLeftMargin}[1]{\parbox[t]{\csllabelwidth}{#1}}
\newcommand{\CSLRightInline}[1]{\parbox[t]{\linewidth - \csllabelwidth}{#1}\break}
\newcommand{\CSLIndent}[1]{\hspace{\cslhangindent}#1}
\ifLuaTeX
  \usepackage{selnolig}  % disable illegal ligatures
\fi

\begin{document}
\maketitle

\hypertarget{this-r-script-is-used-to-clean-up-and-transform-soil-data-to-perform-regression-analysis.}{%
\subsubsection{This R script is used to clean up and transform soil data
to perform regression
analysis.}\label{this-r-script-is-used-to-clean-up-and-transform-soil-data-to-perform-regression-analysis.}}

\hypertarget{overview-of-goals}{%
\section{Overview of Goals}\label{overview-of-goals}}

My research focuses on how soil health and soil microbiology influence
long-term sustainability in semiarid wheat systems. Wheat is one of the
most important crops worldwide, yet the semiarid landscapes where it is
primarily grown are particularly vulnerable to climate change and land
degradation (Asseng et al. 2015; USGCRP 2018). In particular, the
traditional wheat-fallow system practiced throughout the High Plains
ecoregion inefficiently stores water, depletes soil fertility, and has a
high potential for erosion, pushing farmers and researchers to search
for economical ways to build long-term soil health (Norton, Mukhwana,
and Norton 2012; Kaur et al. 2015). Sustaining agriculture in the High
Plains will require alternative farming systems that can rebuild soil
health while remaining profitable (Hansen et al. 2012).

Soil health is critical for sustainable and resilient food production,
yet evaluating soil health in semiarid climates remains challenging.
Agriculture impacts soil health through practices such as crop rotation,
fallowing, tillage, and fertilization, though these effects vary
depending on environmental factors such as climate and soil type. Soil
microorganisms are sensitive indicators of changing soil health, and can
help evaluate the long-term sustainability of land management practices
(Rodgers, Norton, and Diepen 2021). However, there are gaps in our
understanding of how soil microbial properties should be interpreted,
especially in semiarid systems.

This project uses soil data from an experiment on the long-term impacts
of compost application on soil health and soil microbiology in Wyoming
High Plains organic wheat systems. Specifially, four rates of compost
was applied to small plots in a randomized complete block design in
either 2016 or 2021. In contrast to fresh manure or chemical
fertilizers, composted manure is made of concentrated, stabilized
nutrients and carbon that release slowly over time and improve soil
physical properties (Larney et al. 2006). This includes data on soil
physical properties (such as aggregate stability and bulk density), soil
chemical properties (such as total organic matter, labile carbon and
nitrogen pools, and pH) and soil microbiology (enzyme activity,
microbial biomass, and microbial community composition). I aim to use
regression analysis to evaluate the relationships between compost rate
and soil properties. However, many of the variables do not meet the
regression assumptions. Therefore, this project focuses on cleaning and
tidying my dataset, testing the regression assumptions on each variable,
transforming the data to meet those assumptions, and finally visualizing
the data. Ultimately, quantifying soil health benefits over many years
could help growers work towards building the long-term health of their
soil, and could guide efforts to protect soils and create an
agricultural system resistant in the face of climate change and land
degradation.

\hypertarget{load-packages}{%
\subsection{Load Packages}\label{load-packages}}

\begin{Shaded}
\begin{Highlighting}[]
\FunctionTok{library}\NormalTok{(MASS)}
\FunctionTok{library}\NormalTok{(readxl)}
\FunctionTok{library}\NormalTok{(writexl)}
\FunctionTok{library}\NormalTok{(car)}
\FunctionTok{library}\NormalTok{(tidyverse)}
\end{Highlighting}
\end{Shaded}

\hypertarget{import-data}{%
\subsection{Import Data}\label{import-data}}

\begin{Shaded}
\begin{Highlighting}[]
\CommentTok{\#read in data}
\NormalTok{OREI\_soil }\OtherTok{\textless{}{-}} \FunctionTok{read\_excel}\NormalTok{(}\StringTok{"OREI\_05.2021.xlsx"}\NormalTok{, }\AttributeTok{sheet =} \StringTok{"Soil\_Data"}\NormalTok{)}
\NormalTok{OREI\_treatments }\OtherTok{\textless{}{-}} \FunctionTok{read\_excel}\NormalTok{(}\StringTok{"OREI\_05.2021.xlsx"}\NormalTok{, }\AttributeTok{sheet =} \StringTok{"Treatment\_Data\_2021"}\NormalTok{)}
\NormalTok{OREI\_enzymes }\OtherTok{\textless{}{-}} \FunctionTok{read\_excel}\NormalTok{(}\StringTok{"OREI\_05.2021.xlsx"}\NormalTok{, }\AttributeTok{sheet =} \StringTok{"Enzymes"}\NormalTok{)}
\NormalTok{OREI\_PLFA }\OtherTok{\textless{}{-}} \FunctionTok{read\_excel}\NormalTok{(}\StringTok{"OREI\_05.2021.xlsx"}\NormalTok{, }\AttributeTok{sheet =} \StringTok{"PLFAs"}\NormalTok{)}
\end{Highlighting}
\end{Shaded}

\hypertarget{clean-up-data}{%
\subsection{Clean Up Data}\label{clean-up-data}}

\begin{Shaded}
\begin{Highlighting}[]
\CommentTok{\#merge the data}
\NormalTok{OREI\_all }\OtherTok{\textless{}{-}}\NormalTok{ OREI\_treatments }\SpecialCharTok{\%\textgreater{}\%} 
  \FunctionTok{left\_join}\NormalTok{(OREI\_soil, }\AttributeTok{by =} \StringTok{\textquotesingle{}Sample\_ID\textquotesingle{}}\NormalTok{) }\SpecialCharTok{\%\textgreater{}\%} 
  \FunctionTok{left\_join}\NormalTok{(OREI\_enzymes, }\AttributeTok{by =} \StringTok{\textquotesingle{}Sample\_ID\textquotesingle{}}\NormalTok{) }\SpecialCharTok{\%\textgreater{}\%} 
  \FunctionTok{left\_join}\NormalTok{(OREI\_PLFA, }\AttributeTok{by =} \StringTok{\textquotesingle{}Sample\_ID\textquotesingle{}}\NormalTok{)}

\NormalTok{OREI\_all }\OtherTok{\textless{}{-}}\NormalTok{ OREI\_all }\SpecialCharTok{\%\textgreater{}\%} 
\CommentTok{\#remove plots that are fallow or fertilized}
  \FunctionTok{filter}\NormalTok{(Rotation }\SpecialCharTok{==} \StringTok{"wheat"}\NormalTok{, Treatment }\SpecialCharTok{!=} \StringTok{"fertilizer"}\NormalTok{) }\SpecialCharTok{\%\textgreater{}\%} 
  
\CommentTok{\#remove unwanted variables}
\NormalTok{ dplyr}\SpecialCharTok{::}\FunctionTok{select}\NormalTok{(}\SpecialCharTok{{-}}\NormalTok{PER1, }\SpecialCharTok{{-}}\NormalTok{ID, }\SpecialCharTok{{-}}\NormalTok{InorganicC, }\SpecialCharTok{{-}}\NormalTok{Treatment, }\SpecialCharTok{{-}}\NormalTok{Compost\_Rate, }
         \SpecialCharTok{{-}}\NormalTok{Crop, }\SpecialCharTok{{-}}\NormalTok{Rotation, }\SpecialCharTok{{-}}\NormalTok{H2O, }\SpecialCharTok{{-}}\NormalTok{BulkDensity) }

\CommentTok{\#separate into two groups by year of compost application}
\NormalTok{OREI\_2016 }\OtherTok{\textless{}{-}} \FunctionTok{subset}\NormalTok{(OREI\_all, Compost\_Year }\SpecialCharTok{!=} \StringTok{"2020"}\NormalTok{)}
\NormalTok{OREI\_2020 }\OtherTok{\textless{}{-}} \FunctionTok{subset}\NormalTok{(OREI\_all, Compost\_Year }\SpecialCharTok{!=} \StringTok{"2016"}\NormalTok{)}
\end{Highlighting}
\end{Shaded}

\hypertarget{remove-outliers}{%
\subsection{Remove Outliers}\label{remove-outliers}}

Outliers can skew a regression, so first I'll remove any outliers from
each variable using the outlierKB function.

\begin{Shaded}
\begin{Highlighting}[]
\CommentTok{\#I ran this function on each variable in both OREI\_2016 and OREI\_2020 to }
\CommentTok{\#remove outliers.}
\FunctionTok{source}\NormalTok{(}\StringTok{"http://goo.gl/UUyEzD"}\NormalTok{)}
  \CommentTok{\#outlierKD(OREI\_2020, total\_bacteria)}

\CommentTok{\#save no\_outlier data}
  \CommentTok{\#write\_xlsx(OREI\_2020, "OREI\_2020\_no\_outliers.xlsx")}
  \CommentTok{\#write\_xlsx(OREI\_2016, "OREI\_2016\_no\_outliers.xlsx")}

\CommentTok{\#read in no\_outlier data to continue}
\NormalTok{OREI\_2020 }\OtherTok{\textless{}{-}} \FunctionTok{as.data.frame}\NormalTok{(}\FunctionTok{read\_excel}\NormalTok{(}\StringTok{"OREI\_2020\_no\_outliers.xlsx"}\NormalTok{))}
\NormalTok{OREI\_2016 }\OtherTok{\textless{}{-}} \FunctionTok{as.data.frame}\NormalTok{(}\FunctionTok{read\_excel}\NormalTok{(}\StringTok{"OREI\_2016\_no\_outliers.xlsx"}\NormalTok{))}
\end{Highlighting}
\end{Shaded}

I removed outliers from:\\
OREI\_2016: yield, NO3, PMN, DON, MBC, CBH, PHOS, NAG, actino,
gram\_pos, AMF, sapro\_fungi, total\_MB, total\_fungi\\
OREI\_2020: NO3, protein, MBC, MBN, SOC, N, NAG, BX, AG, SUL

\hypertarget{check-regression-assumptions}{%
\section{Check Regression
Assumptions}\label{check-regression-assumptions}}

The four main assumptions of regression analysis are:

\begin{enumerate}
\def\labelenumi{\arabic{enumi}.}
\tightlist
\item
  Observations are independent.
\item
  Normality. The residuals are normally distributed.
\item
  Linearity. The relationship between X and Y is linear.
\item
  Homoscedasticity. The residuals have constant variance for all values
  of X.
\end{enumerate}

\hypertarget{the-samples-come-from-randomized-plots-so-they-are-independent.}{%
\subsection{1. The samples come from randomized plots, so they are
independent.}\label{the-samples-come-from-randomized-plots-so-they-are-independent.}}

\hypertarget{check-for-normality-of-residuals.}{%
\subsection{2. Check for normality of
residuals.}\label{check-for-normality-of-residuals.}}

\begin{Shaded}
\begin{Highlighting}[]
\CommentTok{\#Create an empty list}
\NormalTok{p.vals }\OtherTok{\textless{}{-}} \FunctionTok{list}\NormalTok{()}

\CommentTok{\#This loop runs a linear model on compost \textasciitilde{} each variable, }
\CommentTok{\#evaluates normality with the Shapiro test, and saves the p value.}
\ControlFlowTok{for}\NormalTok{ (i }\ControlFlowTok{in} \FunctionTok{names}\NormalTok{(OREI\_2016[,}\DecValTok{7}\SpecialCharTok{:}\DecValTok{38}\NormalTok{])) \{}
\NormalTok{  mod }\OtherTok{\textless{}{-}} \FunctionTok{lm}\NormalTok{(}\FunctionTok{get}\NormalTok{(i) }\SpecialCharTok{\textasciitilde{}}\NormalTok{ compost, }\AttributeTok{data =}\NormalTok{ OREI\_2016)}
\NormalTok{  p.vals[[i]] }\OtherTok{\textless{}{-}}\NormalTok{ (}\FunctionTok{shapiro.test}\NormalTok{(mod}\SpecialCharTok{$}\NormalTok{residuals))}\SpecialCharTok{$}\NormalTok{p.value \}}
\end{Highlighting}
\end{Shaded}

The following variables have non-normal residuals (p \textgreater{}
0.05):\\
OREI\_2020: DOC, DON, PHOS\\
OREI\_2016: PHOS

\hypertarget{transform-the-non-normal-variables.}{%
\subsubsection{Transform the non-normal
variables.}\label{transform-the-non-normal-variables.}}

\begin{Shaded}
\begin{Highlighting}[]
\CommentTok{\#this function performs a box\_cox transformation on a variable to make it normal}
\NormalTok{box\_cox\_transform }\OtherTok{\textless{}{-}} \ControlFlowTok{function}\NormalTok{(v) \{}
  
\CommentTok{\#calculates the boxcox plot and pulls out lambda}
\NormalTok{  bc }\OtherTok{\textless{}{-}} \FunctionTok{boxcox}\NormalTok{(v }\SpecialCharTok{\textasciitilde{}}\NormalTok{ compost, }\AttributeTok{data =}\NormalTok{ OREI\_2016)}
\NormalTok{  lambda }\OtherTok{\textless{}{-}}\NormalTok{ bc}\SpecialCharTok{$}\NormalTok{x[}\FunctionTok{which.max}\NormalTok{(bc}\SpecialCharTok{$}\NormalTok{y)]}
  
\CommentTok{\#transforms the data using lambda and saves it}
\NormalTok{  v }\OtherTok{\textless{}{-}}\NormalTok{ (v}\SpecialCharTok{\^{}}\NormalTok{lambda}\DecValTok{{-}1}\NormalTok{)}\SpecialCharTok{/}\NormalTok{lambda}
  \FunctionTok{return}\NormalTok{(v) \}}

\CommentTok{\#run box\_cox\_transform on all non{-}normal variables}
\NormalTok{OREI\_2016}\SpecialCharTok{$}\NormalTok{PHOS }\OtherTok{\textless{}{-}} \FunctionTok{box\_cox\_transform}\NormalTok{(OREI\_2016}\SpecialCharTok{$}\NormalTok{PHOS)}
\end{Highlighting}
\end{Shaded}

\begin{Shaded}
\begin{Highlighting}[]
\NormalTok{OREI\_2020}\SpecialCharTok{$}\NormalTok{PHOS }\OtherTok{\textless{}{-}} \FunctionTok{box\_cox\_transform}\NormalTok{(OREI\_2020}\SpecialCharTok{$}\NormalTok{PHOS)}
\end{Highlighting}
\end{Shaded}

\begin{Shaded}
\begin{Highlighting}[]
\NormalTok{OREI\_2020}\SpecialCharTok{$}\NormalTok{DOC }\OtherTok{\textless{}{-}} \FunctionTok{box\_cox\_transform}\NormalTok{(OREI\_2020}\SpecialCharTok{$}\NormalTok{DOC)}
\end{Highlighting}
\end{Shaded}

\begin{Shaded}
\begin{Highlighting}[]
\NormalTok{OREI\_2020}\SpecialCharTok{$}\NormalTok{DON }\OtherTok{\textless{}{-}} \FunctionTok{box\_cox\_transform}\NormalTok{(OREI\_2020}\SpecialCharTok{$}\NormalTok{DON)}
\end{Highlighting}
\end{Shaded}

\begin{Shaded}
\begin{Highlighting}[]
\CommentTok{\#test normality of new data using same loop from above}
\NormalTok{p.vals.trans }\OtherTok{\textless{}{-}} \FunctionTok{list}\NormalTok{()}
\ControlFlowTok{for}\NormalTok{ (i }\ControlFlowTok{in} \FunctionTok{names}\NormalTok{(OREI\_2016[,}\DecValTok{7}\SpecialCharTok{:}\DecValTok{38}\NormalTok{])) \{}
\NormalTok{  mod }\OtherTok{\textless{}{-}} \FunctionTok{lm}\NormalTok{(}\FunctionTok{get}\NormalTok{(i) }\SpecialCharTok{\textasciitilde{}}\NormalTok{ compost, }\AttributeTok{data =}\NormalTok{ OREI\_2016)}
\NormalTok{  p.vals.trans[[i]] }\OtherTok{\textless{}{-}}\NormalTok{ (}\FunctionTok{shapiro.test}\NormalTok{(mod}\SpecialCharTok{$}\NormalTok{residuals))}\SpecialCharTok{$}\NormalTok{p.value \}}
\end{Highlighting}
\end{Shaded}

All the variables are normal now! (p \textgreater{} 0.05)

\hypertarget{check-linearity-using-an-f-test-for-lack-of-fit.}{%
\subsection{3. Check linearity using an F-test for lack of
fit.}\label{check-linearity-using-an-f-test-for-lack-of-fit.}}

\begin{Shaded}
\begin{Highlighting}[]
\CommentTok{\#This function creates a linear and quadratic model for each variable, }
\CommentTok{\#then tests whether the two models differ significantly using ANOVA}
\NormalTok{F\_test }\OtherTok{\textless{}{-}} \ControlFlowTok{function}\NormalTok{(x) \{}
\NormalTok{  mod }\OtherTok{\textless{}{-}} \FunctionTok{lm}\NormalTok{(x }\SpecialCharTok{\textasciitilde{}}\NormalTok{ compost, }\AttributeTok{data =}\NormalTok{ OREI\_2016)}
\NormalTok{  reduced}\OtherTok{\textless{}{-}}\FunctionTok{lm}\NormalTok{(x }\SpecialCharTok{\textasciitilde{}}\NormalTok{ compost, }\AttributeTok{data =}\NormalTok{ OREI\_2016)}
\NormalTok{  full}\OtherTok{\textless{}{-}}\FunctionTok{lm}\NormalTok{(x }\SpecialCharTok{\textasciitilde{}} \FunctionTok{poly}\NormalTok{(compost,}\DecValTok{2}\NormalTok{), }\AttributeTok{data =}\NormalTok{ OREI\_2016)}
  \FunctionTok{return}\NormalTok{(}\FunctionTok{anova}\NormalTok{(reduced, full)}\SpecialCharTok{$}\StringTok{"Pr(\textgreater{}F)"}\NormalTok{) \}}

\CommentTok{\#run this function on each variable}
\NormalTok{F.test}\FloatTok{.2016} \OtherTok{\textless{}{-}} \FunctionTok{list}\NormalTok{()}
\ControlFlowTok{for}\NormalTok{ (i }\ControlFlowTok{in} \FunctionTok{colnames}\NormalTok{(OREI\_2016[,}\DecValTok{7}\SpecialCharTok{:}\DecValTok{38}\NormalTok{])) \{}
\NormalTok{  F.test}\FloatTok{.2016}\NormalTok{[[i]] }\OtherTok{\textless{}{-}} \FunctionTok{F\_test}\NormalTok{(OREI\_2016[[i]]) \}}

\NormalTok{F.test}\FloatTok{.2020} \OtherTok{\textless{}{-}} \FunctionTok{list}\NormalTok{()}
\ControlFlowTok{for}\NormalTok{ (i }\ControlFlowTok{in} \FunctionTok{colnames}\NormalTok{(OREI\_2020[,}\DecValTok{7}\SpecialCharTok{:}\DecValTok{38}\NormalTok{])) \{}
\NormalTok{  F.test}\FloatTok{.2020}\NormalTok{[[i]] }\OtherTok{\textless{}{-}} \FunctionTok{F\_test}\NormalTok{(OREI\_2020[[i]]) \}}
\end{Highlighting}
\end{Shaded}

The regression differs significantly from linear (p\textgreater0.05)
for: OREI\_2020: DOC, protein, porosity, WFPS

\hypertarget{check-constancy-of-residuals-with-the-levene-test.}{%
\subsection{4. Check constancy of residuals with the Levene
test.}\label{check-constancy-of-residuals-with-the-levene-test.}}

\begin{Shaded}
\begin{Highlighting}[]
\CommentTok{\#Run the levene test (from car package) on each variable,}
\CommentTok{\#then save the p values}
\NormalTok{levene}\FloatTok{.2016} \OtherTok{\textless{}{-}} \FunctionTok{list}\NormalTok{()}
\ControlFlowTok{for}\NormalTok{ (i }\ControlFlowTok{in} \FunctionTok{colnames}\NormalTok{(OREI\_2016[,}\DecValTok{7}\SpecialCharTok{:}\DecValTok{38}\NormalTok{])) \{}
\NormalTok{  result }\OtherTok{\textless{}{-}} \FunctionTok{leveneTest}\NormalTok{((OREI\_2016[[i]]) }\SpecialCharTok{\textasciitilde{}} \FunctionTok{as.factor}\NormalTok{(OREI\_2016}\SpecialCharTok{$}\NormalTok{compost))}
\NormalTok{  levene}\FloatTok{.2016}\NormalTok{[[i]] }\OtherTok{\textless{}{-}}\NormalTok{ result}\SpecialCharTok{$}\StringTok{\textasciigrave{}}\AttributeTok{Pr(\textgreater{}F)}\StringTok{\textasciigrave{}}\NormalTok{[}\DecValTok{1}\NormalTok{] \}}
 
\NormalTok{levene}\FloatTok{.2020} \OtherTok{\textless{}{-}} \FunctionTok{list}\NormalTok{()}
\ControlFlowTok{for}\NormalTok{ (i }\ControlFlowTok{in} \FunctionTok{colnames}\NormalTok{(OREI\_2020[,}\DecValTok{7}\SpecialCharTok{:}\DecValTok{38}\NormalTok{])) \{}
\NormalTok{  result }\OtherTok{\textless{}{-}} \FunctionTok{leveneTest}\NormalTok{((OREI\_2020[[i]]) }\SpecialCharTok{\textasciitilde{}} \FunctionTok{as.factor}\NormalTok{(OREI\_2020}\SpecialCharTok{$}\NormalTok{compost))}
\NormalTok{  levene}\FloatTok{.2020}\NormalTok{[[i]] }\OtherTok{\textless{}{-}}\NormalTok{ result}\SpecialCharTok{$}\StringTok{\textasciigrave{}}\AttributeTok{Pr(\textgreater{}F)}\StringTok{\textasciigrave{}}\NormalTok{[}\DecValTok{1}\NormalTok{] \}}
\end{Highlighting}
\end{Shaded}

All data passes the levene test! (p \textless{} 0.05)

\hypertarget{visualize-data}{%
\section{Visualize Data}\label{visualize-data}}

This section needs some work, but for now I'll just print some plots.

\begin{Shaded}
\begin{Highlighting}[]
\NormalTok{p1 }\OtherTok{\textless{}{-}} \FunctionTok{ggplot}\NormalTok{(}\AttributeTok{data=}\NormalTok{ OREI\_2016, }\FunctionTok{aes}\NormalTok{(}\AttributeTok{x =}\NormalTok{ compost, }\AttributeTok{y =}\NormalTok{ DOC))}\SpecialCharTok{+}
  \FunctionTok{geom\_point}\NormalTok{() }\SpecialCharTok{+}
  \FunctionTok{geom\_smooth}\NormalTok{(}\AttributeTok{method =} \StringTok{\textquotesingle{}lm\textquotesingle{}}\NormalTok{) }\SpecialCharTok{+}
  \FunctionTok{labs}\NormalTok{ (}\AttributeTok{title =} \StringTok{\textquotesingle{}One Year After Compost Application\textquotesingle{}}\NormalTok{, }
        \AttributeTok{x =} \StringTok{"Compost Rate (Mg/ha)"}\NormalTok{, }\AttributeTok{y =} \StringTok{"DOC (mg/kg)"}\NormalTok{)}

\NormalTok{p2 }\OtherTok{\textless{}{-}} \FunctionTok{ggplot}\NormalTok{(}\AttributeTok{data=}\NormalTok{ OREI\_2020, }\FunctionTok{aes}\NormalTok{(compost, DOC)) }\SpecialCharTok{+}
  \FunctionTok{geom\_point}\NormalTok{() }\SpecialCharTok{+}
  \FunctionTok{geom\_smooth}\NormalTok{(}\AttributeTok{method=}\StringTok{\textquotesingle{}lm\textquotesingle{}}\NormalTok{) }\SpecialCharTok{+}
  \FunctionTok{labs}\NormalTok{ (}\AttributeTok{title =} \StringTok{\textquotesingle{}Five Years After Compost Application\textquotesingle{}}\NormalTok{, }
        \AttributeTok{x =} \StringTok{"Compost Rate (Mg/ha)"}\NormalTok{, }\AttributeTok{y =} \StringTok{"DOC (mg/kg)"}\NormalTok{)}

\FunctionTok{library}\NormalTok{(patchwork)}
\end{Highlighting}
\end{Shaded}

\begin{verbatim}
## 
## Attaching package: 'patchwork'
\end{verbatim}

\begin{verbatim}
## The following object is masked from 'package:MASS':
## 
##     area
\end{verbatim}

\begin{Shaded}
\begin{Highlighting}[]
\NormalTok{p1 }\SpecialCharTok{+}\NormalTok{ p2 }\SpecialCharTok{+} \FunctionTok{scale\_y\_continuous}\NormalTok{(}\AttributeTok{limits =} \FunctionTok{c}\NormalTok{(}\DecValTok{0}\NormalTok{, }\DecValTok{375}\NormalTok{)) }
\end{Highlighting}
\end{Shaded}

\begin{verbatim}
## `geom_smooth()` using formula 'y ~ x'
\end{verbatim}

\begin{verbatim}
## `geom_smooth()` using formula 'y ~ x'
\end{verbatim}

\includegraphics{Indiv_Project_Prelim_Analyses_files/figure-latex/unnamed-chunk-6-1.pdf}

\hypertarget{references}{%
\section*{References}\label{references}}
\addcontentsline{toc}{section}{References}

\hypertarget{refs}{}
\begin{CSLReferences}{1}{0}
\leavevmode\vadjust pre{\hypertarget{ref-asseng2015rising}{}}%
Asseng, Senthold, Frank Ewert, Pierre Martre, Reimund P Rötter, David B
Lobell, Davide Cammarano, Bruce A Kimball, et al. 2015. {``Rising
Temperatures Reduce Global Wheat Production.''} \emph{Nature Climate
Change} 5 (2): 143--47.

\leavevmode\vadjust pre{\hypertarget{ref-hansen2012}{}}%
Hansen, Neil C, Brett L Allen, R Louis Baumhardt, and Drew J Lyon. 2012.
{``Research Achievements and Adoption of No-till, Dryland Cropping in
the Semi-Arid US Great Plains.''} \emph{Field Crops Research} 132:
196--203.

\leavevmode\vadjust pre{\hypertarget{ref-kaur2015effects}{}}%
Kaur, Gurpreet, Axel Garcia y Garcia, Urszula Norton, Tomas Persson, and
Thijs Kelleners. 2015. {``Effects of Cropping Practices on Water-Use and
Water Productivity of Dryland Winter Wheat in the High Plains Ecoregion
of Wyoming.''} \emph{Journal of Crop Improvement} 29 (5): 491--517.

\leavevmode\vadjust pre{\hypertarget{ref-larney2006}{}}%
Larney, Francis J, Dan M Sullivan, Katherine E Buckley, and Bahman
Eghball. 2006. {``The Role of Composting in Recycling Manure
Nutrients.''} \emph{Canadian Journal of Soil Science} 86 (4): 597--611.

\leavevmode\vadjust pre{\hypertarget{ref-norton2012loss}{}}%
Norton, Jay B, Eusebius J Mukhwana, and Urszula Norton. 2012. {``Loss
and Recovery of Soil Organic Carbon and Nitrogen in a Semiarid
Agroecosystem.''} \emph{Soil Science Society of America Journal} 76 (2):
505--14.

\leavevmode\vadjust pre{\hypertarget{ref-rodgers2021effects}{}}%
Rodgers, Hannah R, Jay B Norton, and Linda TA van Diepen. 2021.
{``Effects of Semiarid Wheat Agriculture Management Practices on Soil
Microbial Properties: A Review.''} \emph{Agronomy} 11 (5): 852.

\leavevmode\vadjust pre{\hypertarget{ref-assessment2018fourth}{}}%
USGCRP, Climate. 2018. {``Fourth National Climate Assessment.''}

\end{CSLReferences}

\end{document}
